\chapter{Conclusion}
\label{conclusion}

The results of this thesis did not provide clear evidence of the effect of the
inter-domain regions' composition on the function, activity, or specificity of human
two-domain PKs.
The proteins could be grouped according to their linker's pI, and three linker types were
defined from the UMAP representation of the quadruples of the averaged physicochemical
attributes of the linkers.
However, no GO term or EC number could characterize the identified protein groups.
A different low-dimensional representation of the linkers' properties was obtained when
a larger set of physicochemical attributes was taken into account.

The presented method relied on the informativeness of the functional annotation services.
Nevertheless, the EC and the GO were found to be too general and too precise,
respectively.
To reduce the detail of the GO, analysis on different GO hierarchy levels was put through,
yet, the identification of unique terms within the clusters has been equally unsuccessful.
As the complete proteome of human two-domain PKs with a single PK domain was examined,
the correlation between the average linker region's properties and the PK activity can be
hereby excluded.

% The insufficiently sized dataset together with the incompleteness and uninformativeness
% of the GO terms and EC numbers contributed to the negative result.

% However, considering the limitations discussed above, a correlation between the
% linker region's properties and the PK activity cannot be excluded yet.
%
% Yet, success cannot be guaranteed.
