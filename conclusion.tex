\chapwithtoc{Conclusion}
\label{conclusion}

Based on the performed research, this thesis did not show any general influence of the
non-domain regions' composition on the performance of human two-domain protein kinases.
The poorly chosen dataset together with the incompleteness and uninformativeness of the
GO terms and EC numbers contributed to the negative result.

However, with respect to the known \emph{open-world assumption}, not observed does not
mean it is not there.
The exposure of the effect of linkers' composition on protein activity could improve our
ability to predict the function of multi-domain proteins, but further research is needed
to disclose the presented problem.
The drafted strategy applied in this thesis could be improved in future work by taking
into account more two-domain protein kinases, and by implementing similarity measures of
GO terms to condensate akin vocables, hence reducing the immensely detailed nature of the
Gene Ontology database.
Yet, success cannot be guaranteed.

% You should summarize what was achieved by the thesis. In a few paragraphs, try to answer the following:
% \begin{itemize}
% \item Was the problem stated in the introduction solved? (Ideally include a list of successfully achieved goals.)
% \item What is the quality of the result? Is the problem solved for good and the mankind does not need to ever think about it again, or just partially improved upon? (Is the incompleteness caused by overwhelming problem complexity that would be out of thesis scope\todo{This is quite common.}, or any theoretical reasons, such as computational hardness?)
% \item Does the result have any practical applications that improve upon something realistic?
% \item Is there any good future development or research direction that could further improve the results of this thesis? (This is often summarized in a separate subsection called `Future work'.)
% \end{itemize}
