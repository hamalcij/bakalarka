\chapter{Conclusion}
\label{conclusion}

Based on the performed research, this thesis did not show any general influence of the
inter-domain regions' composition on the function, activity, or specificity of human
two-domain PKs.
The proteins were divided based on their linker's pI, and three linker types were defined
from the UMAP representation of the quadruples of the averaged physicochemical attributes
of the linkers.
However, no GO term or EC number could characterize the established protein groups.
Furthermore, a more complete set of physicochemical attributes did not recreate the UMAP
representation of the quadruples.

The presented method relied on the informativeness of the functional annotation services.
Nevertheless, the EC and the GO were found to be too general and too precise,
respectively.
To reduce the detail of the GO, analysis on different GO hierarchy levels was put through,
yet, the identification of unique terms within the clusters has been equally unsuccessful.
As the complete proteome of human two-domain PKs with a single PK domain was examined,
the correlation between the average linker region's properties and the PK activity can be
hereby excluded.

% The insufficiently sized dataset together with the incompleteness and uninformativeness
% of the GO terms and EC numbers contributed to the negative result.

% However, considering the limitations discussed above, a correlation between the
% linker region's properties and the PK activity cannot be excluded yet.
%
% Yet, success cannot be guaranteed.
