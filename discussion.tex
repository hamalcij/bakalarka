\chapter{Discussion}
\label{discussion}

The length and amino acid composition of non-domain regions can be crucial for the
regulation of multi-domain proteins' activity, including
PKs~\cite{gogl2019disordered, vigil2004conformational}.
However, to our knowledge, the influence of the linkers' composition on the overall
protein function has not been described \emph{in general} yet.
This thesis tried to address this issue by selecting a dataset of evolutionarily related
multi-domain proteins containing a PK domain, clustering these molecules based on the
physicochemical attributes of their inter-domain regions, and by identifying GO terms and
EC numbers specific to the detected clusters of proteins with various architectures.

Even though it was possible to divide proteins from the studied dataset into three groups
based on the clusters visible in the UMAP representation of the 4-dimensional space of
the linkers' normalized physicochemical characters, no frequent functional annotation
terms could characterize the defined clusters.
There may be several reasons for the lack of success of the designed method.
For example, there may really not be any overall influence of the inter-domain regions'
composition on the function of two-domain PKs.
However, based on the literature search presented above, this proposition seems rather
improbable~\cite{winkler1977tomato, van1997linker, ikebe1998hinge, robinson1998optimizing,
rice1999structural, gokhale2000role, case2000role, pufall2002autoinhibitory,
khalil2008kinesin, hariharan2009insights, smock2010interdomain, liu2010molecular,
shastry2010neck, ma2011dynamic, cyrus2011impact, kalodimos2011nmr, pohane2015modulation,
klement2015effect, rozycki2017length, jakubec2018widespread, gogl2019disordered}.

The narrow dataset may be a significant problem.
In a sample of large size, more information is available; on the other hand, false
presumptions may be concluded from a small sample generating the
observations~\cite{tanaka1987big, hua2005optimal}.
In the studied dataset, 32 different architectures were present within the 117 molecules.
Proteins with the same architecture are frequently related, both evolutionarily and
functionally~\cite{vogel2004structure, hegyi2001annotation, bashton2002geometry}, and the
linkers may introduce new structural features~\cite{papaleo2016role}, thus enhancing the
functionality.
With 3.65625 proteins per architecture on average, no robust conclusion can be made on how
the composition of the linkers affects the behavior of these molecules.

Furthermore, both functional annotation services, GO and EC, have drawbacks regarding
their completeness and homogeneity.
For instance, \citet{gaudet2017gene} claim that the GO is biased and unevenly incomplete.
The GO database is a reflection of primary literature~\cite{gene2004gene}, therefore, less
studied areas are inadequately represented in the GO.
On the other hand, more comprehensively covered parts of the GO are not flawless either,
as they can provide contradictory information which can be caused, for example, by
differences in experimental conditions of comparable
research~\cite{hass2004response, mason2005multiple}.

Within the studied 117 human two-domain PKs, a total of 1,982 GO annotations were fetched
from the UPR.dat file, peaking at 69 terms assigned to a single protein, namely the
\texttt{AAPK1\_HUMAN} PK.
These include even the GO terms from the same hierarchy.
For example, it is possible to reach \texttt{GO:0008610; lipid biosynthetic process} by
recursively applying the transitive relationship ``\texttt{is\_a}'' on
\texttt{GO:0045542; positive regulation of cholesterol biosynthetic process}, which are
both explicitly mentioned in the \texttt{uniprot\_reference\_proteomes.dat} file.
Furthermore, the same protein is classified as \texttt{2.7.11.1 non-specific
serine/threonine protein kinase}; however, at the same time, has the specific EC numbers
\texttt{2.7.11.26}, \texttt{2.7.11.27}, and \texttt{2.7.11.31} assigned to it as well.

To be assigned an EC number, there must be a direct experimental evidence that the
proposed enzyme actually catalyses the claimed reaction~\cite{mcdonald2014fifty}.
60 proteins from the studied dataset were described as \texttt{2.7.11.1 non-specific
serine/threonine protein kinase}, meaning that these PKs are either non-specific,
or their specificity has not been analyzed to
date\footnote{\url{https://www.brenda-enzymes.org/enzyme.php?ecno=2.7.11.1}}.
%Neither of these possibilities are favorable regarding the goal of this thesis.
The strict nature of the EC, as well as the excessiveness of the GO combined with
insufficient amount of research could be overcome by measuring similarities of the GO
terms~\cite{li2010effectively, zhao2018gogo} within the protein clusters instead of
enforcing their absolute exclusion.

Furthermore, it may be desired to employ some more stable method than UMAP to produce a
significant outcome.
The low-dimensional representation generated by UMAP is dependent on the
choice of several
hyperparameters\footnote{\url{https://umap-learn.readthedocs.io/en/latest/parameters.html}}.
Besides, UMAP is a stochastic algorithm, and exact reproduction of the results is only
possible by fixing a random seed
state\footnote{\url{https://umap-learn.readthedocs.io/en/latest/reproducibility.html}}.
The observed linker types are therefore determined by UMAP parametrization.

The elucidation of the effect of the linkers' composition on protein activity could
improve our ability to predict the function of multi-domain proteins, but further
research is needed to disclose the presented problem.
Throughout this thesis, only rough and averaged characteristics of the whole
linkers were considered.
In reality, only a couple of particular residues may have an influence on the protein
activity, the rest of them may be unimportant.
The herein proposed method is not capable of resolving such residues.

% The drafted strategy applied in this thesis could be improved in future work by taking
% into account more two-domain PKs, and by implementing similarity measures of GO terms to
% condensate akin vocables, hence reducing the immensely detailed nature of the GO database.
