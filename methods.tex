\chapter{Methods}
\label{methods}

Protein sequences and MSAs of Pfam domain sequences were obtained from The European
Bioinformatics Institute's
FTP\footnote{\url{ftp://ftp.ebi.ac.uk/pub/databases/Pfam/releases/Pfam32.0/}}.
The archived files \texttt{Pfam-A.full} and \texttt{uniprot\_reference\_proteomes.dat}
(URP.dat) corresponding to Pfam's release 32.0 were downloaded; the MSAs consist of
sequences from the UniProt reference proteomes version 2018\_04.
Protein family annotation software
pfamannot\footnote{\url{https://github.com/hamalcij/pfamannot}} has then been implemented
in C++17 to parse these files and to extract architectures of all proteins containing at
least one \texttt{PF00069} domain, starting and ending positions of all domains present
within such proteins, primary sequences of the whole molecules, and the organisms from
which they originate.
A total of 330,302 such proteins were identified, of which 127,697 were multi-domain
proteins.

To reduce the evolutionary noise, only multi-domain PKs from humans were selected for
further analysis.
A total of 542 multi-domain PKs were identified.
The eukaryotic protein subcellular location predictor
DeepLoc-1.0~\cite{almagro2017deeploc} was then applied to these
proteins\footnote{Sequences spanning more than 6,000 residues were discarded due to
DeepLoc's length limitation.}, and the molecules with their subcellular location
predicted to be the cytoplasm or nucleus were chosen, as the amino acid composition, thus
also the physicochemical properties of membrane proteins differ significantly from their
soluble counterparts~\cite{capaldi1972low, von1988topogenic, tusnady1998principles}.
In addition, membrane proteins have their conformational degrees of freedom reduced due to
their placement in the lipid bilayer, and are therefore not suitable for the purpose of
this thesis.
Two-domain proteins with a single PK domain were then selected from the emerging 244
cytosolic or nuclear human multi-domain PKs, yielding the final dataset containing a
total of 117 protein molecules.

Two sets of physicochemical properties were then acquired for the inter-domain regions in
the final dataset.
The first set, which was in the spotlight of this study, consists of quadruples composed
of the logarithm of the linker's length, the linker's pI, the percentage
of charged amino acids in the linker, and the linker's GRAVY index~\cite{kyte1982simple}.
The pI and the GRAVY index roughly describe the linker's electric charge and its
hydrophobicity, respectively.
These values were calculated with the ExPASy's ProtParam tool~\cite{gasteiger2005protein},
and their distributions across the whole dataset were computed.
In the second set, each linker was described using the 553 physicochemical and
biochemical properties of amino acids from the AAindex database, version
9.2~\cite{nakai1988cluster, tomii1996analysis, kawashima1999aaindex, kawashima2000aaindex,
kawashima2007aaindex}.
A dimensionality reduction technique Uniform Manifold Approximation and Projection
(UMAP)~\cite{mcinnes2018umap} was then applied to various normalized subsets of both
sets, thus producing two-dimensional (2D) representations of the feature spaces aminable
to a visual analysis resulting in, for example, the recognition of cluster structure in
the data.

The EC and GO terms associated with proteins in the URP.dat file were parsed.
The former describes the enzyme specificity~\cite{webb1992enzyme}, the latter terms
biological processes, cellular components, and molecular functions of genes and their
products~\cite{ashburner2000gene, gene2019gene}.
The GO is a weakly hierarchical vocabulary~\cite{ashburner2000gene}, and the hierarchy
level of GO terms in the URP.dat file is not standardized.
Therefore, for each protein, parents of its terms were recursively obtained and merged
from the \texttt{go-basic.obo}
file\footnote{\url{http://current.geneontology.org/ontology/go-basic.obo}}, release
2020-06-01, thus enabling cluster analysis on different GO levels.
GO terms labeled as obsolete were excluded.
The EC and GO terms were then embedded into the UMAP projections and the presence of
unique terms within the identified clusters was examined.
The same process was executed on the density clusters (see below) as well.
The results were visualized using the Python graphics package
Matplotlib~\cite{hunter2007matplotlib}, version 1.5.1.
